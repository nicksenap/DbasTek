\documentclass[a4paper]{article}
\usepackage[english]{babel}
\usepackage{amsmath}
\usepackage{graphicx}
\usepackage{color}   %May be necessary if you want to color links
\usepackage{hyperref}
\title{DD1368 Database Technology Lecture notes}
\author{Zixian Song}

\hypersetup{
    colorlinks=true, %set true if you want colored links
    linktoc=all,     %set to all if you want both sections and subsections linked
    linkcolor=black,  %choose some color if you want links to stand out
}

\begin{document}
\maketitle

\begin{figure}[b] 
\centering 
\includegraphics[width=2.045cm,height=2.335cm]{kth_svv_comp_science.eps}
\end{figure}
\newpage
\tableofcontents
\newpage

\section{Lecture 1: Intro}
2016-01-20 \\
    Introduction
    \subsection{Content of databases}
        \begin{itemize}
            \item Design of databases 
            \item E/R  model, relation model, semistructured model, XML
  
        \item Database programing 
        \item SQL, XPath, XQuery 
        \item Relational algebra, Tuple Calculus, FOL. 
        \item ODBC JDBC or etc \ldots
        \end{itemize}
    \subsection{Interesing Stuff About Database}
        \begin{itemize}
            \item Websearch 
            \item Data mining 
            \item Scientific and medical databases 
            \item integrationg information. 
            \item \textbf{NSA! wikileaks HeathCare.gov} 
            \item Still more. \ldots
        \end{itemize}
    \subsection{And More}
        Database often have unique 
        concurrency-control problems. \\
        \begin{itemize}
            \item Many activities (transactions) at the database at all times.
            \item Must not confuse actions, e.g. two withdrewal from the same account must each debit the account.
            \item ATC :
        \end{itemize}
    \subsection{What is a Data Model}
    \begin{enumerate}
        \item Mathematical represention of data.
        \begin{itemize}
            \item Examples:relational model = tables;
            \item Semistruced model = trees,graphs
        \end{itemize}
        \item Operations on data
        \item Constraints
    \end{enumerate}
    \subsection{Schema}
        \begin{itemize}
            \item Relation schema = relation name and attribute list.
        \end{itemize}

    \subsection{Our Running Example}
    Beer(\underline{name}, manf) \\
    Bars(\underline{name}, addr, license) \\
    Drinkers(\underline{name}, addr, phone) 
    \\
    Underline = \textit{key} (tuples cannot have the same value in all key attributes).
        \begin{itemize}
         \item Excellent 
         \end{itemize}
    \subsection{Example: Multiattribute Key}
    \begin{verbatim}
        CREATE TABLE Sells (
            bar CHAR(20),
            beer VARCHAR(20),
            price REAL,
            PRIMARY KEY (bar, beer) 
            );
    \end{verbatim}
    \subsection{Semistrucured Data}
        \begin{itemize}
            \item Another data model, based on trees graphs.
            \item \textbf{Motivation}: flexible represention of data
            \item \textbf{Motivation}: sharing of \textit{documents} among systems and databases
        \end{itemize}


    \newpage

    \section{Lecture 2: Relational Algebra and Tuple Calculus}
        \subsection{Ideal Picture}
            \begin{itemize}
                \item Tuple Calculus: Declarative(logical) expression of what the user wants
                \item Relational Algebra: Procedural Expression that can obtain answers
                \item Opetimized Relational Algebra: Can be efficienty executed over database
            \end{itemize}
        \subsection{Example}
        "Names of companies in Sweden that have a supplier from China"
            \begin{verbatim}
                {x.name | company(x) AND x.country = 'Sweden' AND
                (Ey)(Ex)( supplies(y) AND company(z) AND 
                z.contry = 'China' AND x.id = y.to AND y.from = z.id)}
            \end{verbatim}
        \subsection{What is Relational Algebra}
                \begin{itemize}
                    \item An algebra whose operands are relations or variables that represent relations.
                    \item Operators are designed to do the most common things that we need to do with relations in a database.
                    \item The result is an algebra that can be used as a query language for relations.
                
                \end{itemize}
        \subsection{Core Relational Algebra}
            \begin{itemize}
                \item Union, intersection, and difference.
                    \begin{itemize}
                        \item Usual set operations, but \textit{both operands must have the same relation schema.}
                    \end{itemize}
                \item \textbf{Selection}: picking certain rows.
                \item \textbf{Projection}: picking certain columns.
                \item \textbf{Products} and \textbf{joins}: compositions of relations.
                \item \textbf{Renaming} of relations and attributes.
            \end{itemize}

        \subsection{Selection}
        \begin{center}
            $R1:= \sigma_{c}(R)$
        \end{center}
            \begin{itemize}
                \item $C$ is a condition (as in “if” statements) that refers to attributes 
of R2.
                \item R1 is all those tuples of R2 that satisfy 
$C$.
            \end{itemize}
        \subsection{Selection:Example}

            Relation Sells:
            \begin{center}
            \begin{tabular}{ |c|c|c| } 
            \hline
                bar & beer & price \\ 
            \hline
                Joe’s & Bud & 2.50 \\ 
                Joe’s & Miller & 2.75 \\ 
                Sue’s & Bud & 2.50 \\
                Sue’s & Miller & 3.00 \\
            \hline
            \end{tabular}
            \\ 
            $JoeMenu:= \sigma_{bar} = "joe's"(Sells):$ 
            \\ 
            
            \begin{tabular}{ |c|c|c| } 
            \hline
                bar & beer & price \\ 
            \hline
                Joe’s & Bud & 2.50 \\ 
                Joe’s & Miller & 2.75 \\ 
            \hline
            \end{tabular}   
            \end{center}



        \subsection{Projection}
            \begin{center}
            $R2:= \pi_{L}(R):$
            \end{center}
                \begin{itemize}
                    \item $L$ is a list of attributes from the schema of R2.
                    \item R1 is constructed by looking at each tuple of R2, extracting the attributes on list $L$, in the order specified, and creating from those components a tuple for R1.
                    \item Eliminate duplicate tuples, if any.
                \end{itemize}
        \subsection{Product}
            \begin{center}
                $R3:= R1 \times R2$
            \end{center}
            \begin{itemize}
                \item Pair each tuple t1 of R1 with each tuple t2 of R2.
                \item Concatenation t1 t2 is a tuple of R3.
                \item Schema of R3 is the attributes of R1 and then R2, in order.
                \item But beware attribute $A$ of the same name in R1 and R2: use R1.$A$ and R2.$A$.
            \end{itemize}
        \subsection{example}
            Names if companis who supply another company: \\
            $ \{x.name | company(x) \wedge (\exists y)(Supplies(Y) \wedge y.from = 
            x.name)\}$

                


    

\end{document}
